\documentclass[11pt,a4paper]{article}

% ---- Idioma y tipografía
\usepackage[spanish, es-noquoting, es-noshorthands]{babel}
\usepackage[T1]{fontenc}
\usepackage[utf8]{inputenc} % eliminar si usas XeLaTeX/LuaLaTeX
\usepackage{lmodern}
\usepackage{microtype}
\usepackage{float}
% ---- Matemáticas
\usepackage{amsmath}
\usepackage{amssymb}

% ---- Márgenes y diseño
\usepackage[a4paper,margin=2.5cm]{geometry}
\usepackage{parskip}

% ---- Enlaces
\usepackage[hidelinks]{hyperref}
\usepackage{bookmark}

% ---- Cabeceras y pies
\usepackage{fancyhdr}
\pagestyle{fancy}
\fancyhf{}
\renewcommand{\headrulewidth}{0.4pt}
\lhead{\textsc{\asignatura}}
\rhead{\textbf{\tema}}
\cfoot{\thepage}

% ---- Listas y símbolos
\usepackage{enumitem}
\setlist{itemsep=0.3em, topsep=0.5em}

% ---- Iconos y color
\usepackage{fontawesome5}
\usepackage{xcolor}
\usepackage[most]{tcolorbox}
\tcbuselibrary{breakable,skins}

% ---- Figuras
\usepackage{graphicx}
\usepackage{caption}

% ---- Colores propios
\definecolor{uniPrimary}{HTML}{0A7AC3}
\definecolor{uniSoft}{HTML}{E9F4FB}
\definecolor{uniAccent}{HTML}{F5B700}
\definecolor{uniOK}{HTML}{2E7D32}
\definecolor{uniWarn}{HTML}{C62828}

% ---- Metadatos reutilizables
\newcommand{\asignatura}{Seguridad de Sistemas Informáticos}
\newcommand{\tema}{Tema 3 — Software Malicioso}
\newcommand{\clase}{Clase 1}

% ---- Estética de secciones
\usepackage{titlesec}
\titleformat{\section}{\Large\bfseries\color{uniPrimary}}{\thesection}{0.6em}{}
\titleformat{\subsection}{\large\bfseries}{\thesubsection}{0.5em}{}
\titleformat{\subsubsection}{\bfseries}{\thesubsubsection}{0.5em}{}

% ---- Cajas reutilizables
\tcbset{
    commonstyle/.style={
        enhanced, breakable,
        left=8pt,right=8pt,top=8pt,bottom=8pt,
        boxrule=0.8pt,
        fonttitle=\bfseries
    }
}

\newtcolorbox{ObjetivosBox}[1][]{
    commonstyle,
    title={\faBullseye\; Índice},
    colback=uniSoft, colframe=uniPrimary,#1
}

\newtcolorbox{DefBox}[1][]{
    commonstyle,
    title={\faBook\; Definición},
    colback=white, colframe=uniAccent,#1
}

\newtcolorbox{NotaBox}[1][]{
    commonstyle,
    title={\faStickyNote\; Nota},
    colback=white, colframe=uniPrimary,#1
}

\newtcolorbox{RecordatorioBox}[1][]{
    commonstyle,
    title={\faBell\; Recordatorio},
    colback=uniAccent!15, colframe=uniAccent,#1
}

\newtcolorbox{ChecklistBox}[1][]{
    commonstyle,
    title={\faTasks\; Tareas / Checklist},
    colback=white, colframe=uniPrimary,#1
}

\newtcolorbox{ResumenBox}[1][]{
    commonstyle,
    title={\faHighlighter\; Resumen rápido (5 líneas)},
    colback=uniSoft, colframe=uniPrimary,#1
}

\newtcolorbox{VocabBox}[1][]{
    commonstyle,
    title={\faLanguage\; Vocabulario clave},
    colback=white, colframe=uniPrimary,#1
}

% =========================================================
\begin{document}

\tableofcontents
\newpage

\section{Malware: Conceptos básicos}

\begin{DefBox}
    \textbf{Malware} (software malicioso) Un programa que se inserta en un sistema, generalmente de una forma encubierta, con el objetivo de comprometer la triada CIA: Confidencialidad, Integridad y Disponibilidad.
\end{DefBox}

\subsection{Terminología}

\begin{table}[H]
\centering
\small
\begin{tabular}{|p{4cm}|p{10cm}|}
\hline
\textbf{Nombre} & \textbf{Descripción resumida} \\ \hline
Amenaza persistente avanzada (APT) & Ataques prolongados a objetivos estratégicos mediante múltiples técnicas y malware, usualmente patrocinados por el Estado. \\ \hline
Software publicitario (Adware) & Publicidad integrada en software que genera anuncios o redirecciones. \\ \hline
Equipo de ataque (Attack kit) & Conjunto de herramientas para crear y propagar nuevo malware automáticamente. \\ \hline
Raíz automática (Auto-rooter) & Herramientas que permiten acceso remoto a máquinas comprometidas. \\ \hline
Puerta trasera (Backdoor, Trapdoor) & Mecanismo que evita controles de seguridad y permite acceso no autorizado. \\ \hline
Descargadores (Downloaders) & Código que instala otros programas maliciosos en un sistema comprometido. \\ \hline
Descarga automática (Drive-by-download) & Código en un sitio web que explota vulnerabilidades al ser visitado. \\ \hline
Exploits & Código diseñado para aprovechar vulnerabilidades específicas. \\ \hline
Inundaciones (Flooders; DoS client) & Generan grandes volúmenes de tráfico para causar denegación de servicio. \\ \hline
Registradores de teclas (Keyloggers) & Capturan las pulsaciones del teclado en un sistema. \\ \hline
Bomba lógica (Logic bomb) & Código que se activa al cumplirse una condición, ejecutando una carga maliciosa. \\ \hline
Macrovirus & Virus basado en macros de documentos que se ejecuta al abrir o editar archivos. \\ \hline
Código móvil & Script o macro que puede ejecutarse en distintas plataformas sin modificación. \\ \hline
Rootkit & Herramientas que ocultan el control del sistema tras obtener acceso de nivel raíz. \\ \hline
Programas Spammers & Software que envía grandes volúmenes de correos no deseados. \\ \hline
Software espía (Spyware) & Programa que recopila y transmite información confidencial del usuario. \\ \hline
Caballo de Troya (Trojan horse) & Programa aparentemente útil que contiene funciones ocultas maliciosas. \\ \hline
Virus & Malware que se replica al ejecutarse en otros archivos o programas. \\ \hline
Gusano (Worm) & Programa que se propaga automáticamente a través de redes. \\ \hline
Zombi, robot (Bot) & Programa en un equipo infectado que ejecuta órdenes para atacar otros sistemas. \\ \hline
\end{tabular}
\caption{Tipos de amenazas y malware con descripción resumida.}
\end{table}

\section{Tipos de Software Maliciosos}

\subsection{Clasificación del Malware}

Se puede clasificar en función de:
\begin{itemize}
    \item \textbf{Modo de propagación:} cómo se distribuye e infecta.
    \item \textbf{Tipo de carga (payload) maliciosa:} acciones que realiza una vez dentro del sistema.
\end{itemize}

Los mecanismos de propagación incluyen:
\begin{itemize}
    \item Infección de contenido existentes
    \item Explotación de vulnerabilidades
    \item Ingeniería social
\end{itemize}

Los mecanismos de acción incluyen:
\begin{itemize}
    \item Corrupción de datos.
    \item Robo de servicios o convertir en zombi.
    \item Robo de información, registro de teclas.
    \item Sigilo o ocultar su presencia.
\end{itemize}

\subsection{Attack Kits}

Inicialmente, el desarrollo e implementación de malware requería habilidad técnica considerable.

A principios de los 2000 surgieron herramientas y kits que automatizaban la creación y distribución de malware, facilitando su uso a atacantes menos experimentados.

Las variantes de malware que se pueden generar implican un problema importante para los que defienden los sistemas.

\subsection{Attack Sources}

Otro avance es la fuente de ataques, ha pasado de un único atacante a redes de atacantes coordinados. Con objetivos lucrativos o políticos.

\subsection{Amenazas Persistentes Avanzadas (APTs)}

Una APT es una aplicación persistente y con buenos recuersos de una amplia variedad de tecnologías de intrusión.

Se caracterizan por su amplia persistencia y sigilo. Normalmente atribuidas a organizaciones estatales o criminales.

\textbf{Características del APT:}
\begin{itemize}
    \item \textbf{Persistente} Aplicación dedicada de los ataques durante un periodo prolongado.
    \item \textbf{Avanzada} Uso de múltiples técnicas y herramientas para lograr el objetivo. Los componentes se seleccionan cuidadosamente para lograr un objetivo determinado designado especialmente para el ataque.
\end{itemize}

\textbf{Ataques APT:}



\section{Virus}

\section{Tipos de propagaciones y vulnerabilidades}

\subsection{Descubrimiento de targets, objetivos}

\subsection{Ataques basados en ingeniería social}

\section{Contramedidas contra el malware}

La solución más efectiva contra el malware es la \textbf{prevención}.

Cuatro elementos principales de prevención:
\begin{itemize}
    \item \textbf{Política}
    \item \textbf{Conciencia}
    \item \textbf{Mitigación de vulnerabilidades}
    \item \textbf{Mitigación de amenazas}
\end{itemize}

Si la prevención falla se pueden usar mecanismos técnicos para respaldar las siguientes opciones de mitigación:
\begin{itemize}
    \item \textbf{Detección}
    \item \textbf{Identificación}
    \item \textbf{Eliminación}
\end{itemize}

\subsection{Generaciones de software antivirus}

\begin{itemize}
    \item \textbf{Primera generación:} Escaneres simples. Limitado a malware conocido. Firma hash de malware.
    \item \textbf{Segunda generación:} Escáneres heuristicos.
    \item \textbf{Tercera generación:} Trampas de actividad.
    \item \textbf{Cuarta generación:} Protección con todas las funciones.
\end{itemize}

\subsection{Análisis de zona de pruebas (Sandbox)}

Ejecutar código potencialmente malicioso en un entorno limitado emulado o en una maquina virtual.

Permite que el código se ejecute en un entorno limitado y controlado donde su compotramiento puede monitorizarse sin riesgo para el sistema anfitrión.

\subsection{Análisis dinámico de malware}

Se integra con el sistema operativo de una computadora y monitoriza el comportamiento del malware en riempo real.

Denido a que el código malicioso debe ejecutarse para observar su comportamiento, el análisis dinámico es más efectivo para detectar malware desconocido o polimórfico.

\subsection{Enfoques de escaneo perimetral}

El software antivirus generalmente se incluye en los servicios de proxy web y de correo electrónico ejecuta un firewall de una organización

También se pueden implementar sistemas de detección y prevención de intrusiones (IDS/IPS) en el perímetro de la red para identificar y bloquear tráfico malicioso.

El enfoque se limita a escanear software.

\end{document}
