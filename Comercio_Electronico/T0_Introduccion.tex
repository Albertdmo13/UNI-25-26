\documentclass[11pt,a4paper]{article}

% ---- Idioma y tipografía
\usepackage[spanish]{babel}
\usepackage[T1]{fontenc}
\usepackage[utf8]{inputenc}
\usepackage{lmodern}
\usepackage{microtype}

% ---- Márgenes y diseño
\usepackage[a4paper,margin=2.2cm]{geometry}
\usepackage{parskip}

% ---- Colores y cajas
\usepackage[most]{tcolorbox}
\usepackage{xcolor}
\usepackage{enumitem}
\usepackage{fontawesome5}

% ---- Colores propios
\definecolor{uniPrimary}{HTML}{0A7AC3}
\definecolor{uniSoft}{HTML}{E9F4FB}
\definecolor{uniAccent}{HTML}{F5B700}

% ---- Cajas útiles
\newtcolorbox{ObjetivosBox}{
	title={\faBullseye\; Objetivos},
	colback=uniSoft,
	colframe=uniPrimary,
	enhanced, breakable,
	left=8pt,right=8pt,top=8pt,bottom=8pt,
	boxrule=0.8pt,
	fonttitle=\bfseries
}

\newtcolorbox{TemarioBox}{
	title={\faBullseye\; Temario},
	colback=uniSoft,
	colframe=uniPrimary,
	enhanced, breakable,
	left=8pt,right=8pt,top=8pt,bottom=8pt,
	boxrule=0.8pt,
	fonttitle=\bfseries
}

\newtcolorbox{NotaBox}{
	title={\faStickyNote\; Nota},
	colback=white,
	colframe=uniPrimary,
	enhanced, breakable,
	left=8pt,right=8pt,top=8pt,bottom=8pt,
	boxrule=0.8pt,
	fonttitle=\bfseries
}

\newtcolorbox{RecordatorioBox}{
	title={\faBell\; Recordatorio},
	colback=uniAccent!15,
	colframe=uniAccent,
	enhanced, breakable,
	left=8pt,right=8pt,top=8pt,bottom=8pt,
	boxrule=0.8pt,
	fonttitle=\bfseries
}

% ---- Documento
\begin{document}

	% Portada rápida
	\begin{center}
		{\huge \textbf{Comercio Electrónico}} \\[0.5cm]
		{\Large Universidad \;--\; Curso 2025} \\[0.3cm]
		\rule{0.8\linewidth}{0.5pt} \\[0.3cm]
		{\faUser\; Nombre del alumno: Alberto Díaz} \\[0.2cm]
		{\faChalkboardTeacher\; Profesor: Vanesa Herrera} \\[1.2cm]
	\end{center}

	\section*{Introducción}

	\begin{TemarioBox}
		\begin{itemize}[leftmargin=1.5em]
			\item Tema 1. Introducción
			\item Tema 2. Infraestructura para COE
			\item Tema 3. Marketing y posicionamiento web
			\item Tema 4 + Tema 5. Confianza Seguridad y Legislación.
		\end{itemize}
	\end{TemarioBox}
	\begin{ObjetivosBox}
		\textbf{Aprendizaje Basado en Proyectos (PBL) -- Project Based Learning}
		\begin{itemize}[leftmargin=1.5em]
			\item Planteamiento de un problema real como punto de partida.
			\item Investigación, análisis y búsqueda de soluciones innovadoras.
			\item Desarrollo del pensamiento crítico y habilidades de resolución de problemas.
			\item Trabajo colaborativo y gestión de equipos.
			\item Enfoque multidisciplinar para abordar los retos del proyecto.
		\end{itemize}

		\vspace{0.5em}
		\textbf{Aprendizaje-Servicio (ApS):}
		\begin{itemize}[leftmargin=1.5em]
			\item Realización de un servicio solidario vinculado al entorno.
			\item Aplicación práctica de los conocimientos adquiridos.
			\item Fomento del liderazgo, la comunicación y el compromiso social.
			\item Desarrollo de competencias a través del trabajo de campo.
		\end{itemize}

		\textbf{Juntando las piezas:}
		\begin{itemize}[leftmargin=1.5em]
			\item Problema real
			\item Investigación
			\item Pensamiento crítico
			\item Servicio solidario
			\item Resolución de problemas
			\item Liderazgo
			\item Trabajo en grupo
			\item Multidisciplinar
			\item Comunicación
			\item Compromiso
			\item Trabajo de campo
		\end{itemize}
		\textbf{IMPORTANCIA DEL TRABAJO EN EQUIPO}
	\end{ObjetivosBox}

	\subsection*{Evaluación}

	\begin{center}
		\renewcommand{\arraystretch}{1.4} % más espacio entre filas
		\begin{tabular}{|l|c|c|c|}
			\hline
			\textbf{Criterio} & \textbf{Porcentaje} & \textbf{Obligatorio} & \textbf{Recuperable} \\
			\hline
			Teoría (25\% cada parcial) & 50\% & Sí & Sí \\
			Práctica & 25\% & Sí & Sí \\
			Aprovechamiento en clase & 10\% & No & No \\
			Trabajo teórico & 15\% & No & Sí \\
			\hline
			\textbf{Total} & \textbf{100\%} & & \\
			\hline
		\end{tabular}
	\end{center}

\end{document}
