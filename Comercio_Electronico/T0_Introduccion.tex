\documentclass[11pt,a4paper]{article}

% ---- Idioma y tipografía
\usepackage[spanish]{babel}
\usepackage[T1]{fontenc}
\usepackage[utf8]{inputenc}
\usepackage{lmodern}
\usepackage{microtype}

% ---- Márgenes y diseño
\usepackage[a4paper,margin=2.2cm]{geometry}
\usepackage{parskip}

% ---- Colores y cajas
\usepackage[most]{tcolorbox}
\usepackage{xcolor}
\usepackage{enumitem}
\usepackage{fontawesome5}

% ---- Colores propios
\definecolor{uniPrimary}{HTML}{0A7AC3}
\definecolor{uniSoft}{HTML}{E9F4FB}
\definecolor{uniAccent}{HTML}{F5B700}

% ---- Cajas útiles
\newtcolorbox{ObjetivosBox}{
	title={\faBullseye\; Objetivos},
	colback=uniSoft,
	colframe=uniPrimary,
	enhanced, breakable,
	left=8pt,right=8pt,top=8pt,bottom=8pt,
	boxrule=0.8pt,
	fonttitle=\bfseries
}

\newtcolorbox{TemarioBox}{
	title={\faBook\; Temario},
	colback=uniSoft,
	colframe=uniPrimary,
	enhanced, breakable,
	left=8pt,right=8pt,top=8pt,bottom=8pt,
	boxrule=0.8pt,
	fonttitle=\bfseries
}

\newtcolorbox{NotaBox}{
	title={\faStickyNote\; Nota},
	colback=white,
	colframe=uniPrimary,
	enhanced, breakable,
	left=8pt,right=8pt,top=8pt,bottom=8pt,
	boxrule=0.8pt,
	fonttitle=\bfseries
}

\newtcolorbox{RecordatorioBox}{
	title={\faBell\; Recordatorio},
	colback=uniAccent!15,
	colframe=uniAccent,
	enhanced, breakable,
	left=8pt,right=8pt,top=8pt,bottom=8pt,
	boxrule=0.8pt,
	fonttitle=\bfseries
}

% ---- Documento
\begin{document}

	% Portada mejorada
	\begin{center}
		\centering
		\vspace*{2cm}
		{\huge \bfseries Comercio Electrónico \par}
		\vspace{0.7cm}
		{\Large Universidad de Castilla-La Mancha -- Curso 2025 \par}
		\vspace{0.5cm}
		\rule{0.8\linewidth}{0.8pt} \par
		\vspace{0.5cm}
		{\faUser\; \textbf{Alumno:} Alberto Díaz \par}
		\vspace{0.2cm}
		{\faChalkboardTeacher\; \textbf{Profesora:} Vanesa Herrera Tirado \par}
		\vspace{0.2cm}
		{\faEnvelope\; \texttt{vanesa.herrera@uclm.es} \par}
		\vfill
	\end{center}

	\section*{Profesorado}
	\begin{itemize}[leftmargin=1.5em]
		\item \textbf{Profesora}: Vanesa Herrera Tirado
		\item Grupo de investigación AIR:
		\item Contratada a cargo de proyectos de investigación
		\item Experiencia en empresas, máster y docencia
	\end{itemize}

	\section*{Horario de Tutorías}
	\begin{itemize}[leftmargin=1.5em]
		\item \textbf{Lunes}: 9:30\,h -- 12:30\,h
		\item \textbf{Jueves}: 9:30\,h -- 12:30\,h
	\end{itemize}

	\section*{Horario de Clase}
	\begin{itemize}[leftmargin=1.5em]
		\item \textbf{Lunes} (Laboratorio): 18:30\,h -- 19:50\,h
		\item \textbf{Martes} (Teoría, Aula 2.2 Grace Murray):
		\begin{itemize}
			\item 17:00\,h -- 18:20\,h
			\item 18:30\,h -- 19:50\,h
		\end{itemize}
	\end{itemize}

	\section*{Introducción}
	\begin{TemarioBox}
		\begin{itemize}[leftmargin=1.5em]
			\item Tema 1. Introducción
			\item Tema 2. Infraestructura para COE
			\item Tema 3. Marketing y posicionamiento web
			\item Tema 4. Confianza y seguridad en COE
			\item Tema 5. Legislación aplicable a COE
		\end{itemize}
	\end{TemarioBox}

	\begin{ObjetivosBox}
		\textbf{Competencias generales (BOE 4/8/2009):}\\
		Capacidad de concebir sistemas, aplicaciones y servicios basados en tecnologías de red (Internet, web, comercio electrónico, multimedia, servicios interactivos y computación móvil).

		\vspace{0.7em}
		\textbf{Aprendizaje Basado en Proyectos (PBL):}
		\begin{itemize}[leftmargin=1.5em]
			\item Problema complejo/real como punto de partida.
			\item Investigación, análisis y pensamiento crítico.
			\item Trabajo en grupo y multidisciplinar.
		\end{itemize}

		\vspace{0.7em}
		\textbf{Aprendizaje-Servicio (ApS):}
		\begin{itemize}[leftmargin=1.5em]
			\item Servicio solidario vinculado al entorno.
			\item Resolución de problemas y liderazgo.
			\item Comunicación, compromiso y trabajo de campo.
		\end{itemize}

		\vspace{0.7em}
		\textbf{Juntando las piezas:}\\
		Problema real, investigación, pensamiento crítico, trabajo en grupo, multidisciplinar, servicio solidario, resolución de problemas, liderazgo, comunicación, compromiso y trabajo de campo.

		\vspace{0.7em}
		\textbf{IMPORTANCIA DEL TRABAJO EN EQUIPO}.
	\end{ObjetivosBox}

	\subsection*{Evaluación}
	\begin{center}
		\renewcommand{\arraystretch}{1.4}
		\begin{tabular}{|p{5.5cm}|c|c|c|}
			\hline
			\textbf{Criterio} & \textbf{Puntos / \%} & \textbf{Obligatorio} & \textbf{Recuperable} \\
			\hline
			Pruebas de progreso (P1--P3) & 5 (50\%) & Sí & Sí \\
			Interacción en clase & 1 (10\%) & No & No \\
			Trabajo teórico ind. (Portfolio) & 1,5 (15\%) & No & Sí \\
			Trabajo en grupo (proyecto real) & 2,5 (25\%) & Sí & Sí \\
			\hline
			\textbf{Total} & \textbf{10 (100\%)} & & \\
			\hline
		\end{tabular}
	\end{center}

	\begin{NotaBox}
		\begin{itemize}[leftmargin=1.5em]
			\item Las pruebas de progreso son acumulativas: \textbf{P1} (0,75 ptos), \textbf{P2} (1,5 ptos), \textbf{P3} (2,75 ptos).
			\item El trabajo en grupo incluye diseño y desarrollo de un problema real de COE, con reparto autónomo de la nota entre los integrantes.
			\item El trabajo teórico individual consiste en una solución web (el producto sois vosotros).
			\item Los puntos del trabajo en grupo se reparten por nosotros.
		\end{itemize}
	\end{NotaBox}

	\begin{RecordatorioBox}
		\textbf{Sistema de tickets:} Se obtienen cumpliendo actividades y pueden canjearse por:
		\begin{itemize}[leftmargin=1.5em]
			\item Pasar pregunta tipo test
			\item Ayuda en pregunta de examen
			\item Tiempo extra de entrega
			\item Pista para pregunta de estudio/examen
			\item Tiempo extra en examen
			\item Turno prioritario
		\end{itemize}
	\end{RecordatorioBox}

	\begin{ObjetivosBox}
		\textbf{Trabajo Práctico en Grupo (TPG):} \\
		\begin{itemize}[leftmargin=1.5em]
			\item 25\% de la asignatura.
			\item Desarrollar una solución de comercio electrónico real para una PYME local.
			\item Grupos de 5 personas.
		\end{itemize}
	\end{ObjetivosBox}

\end{document}
