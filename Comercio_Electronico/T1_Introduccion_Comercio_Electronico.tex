\documentclass[11pt,a4paper]{article}

% ---- Idioma y tipografía
\usepackage[spanish]{babel}
\usepackage[T1]{fontenc}
\usepackage[utf8]{inputenc}
\usepackage{lmodern}
\usepackage{microtype}

% ---- Márgenes y diseño
\usepackage[a4paper,margin=2.2cm]{geometry}
\usepackage{parskip}

% ---- Enlaces
\usepackage[hidelinks]{hyperref}
\usepackage{bookmark}

% ---- Cabeceras y pies
\usepackage{fancyhdr}
\pagestyle{fancy}
\fancyhf{}
\renewcommand{\headrulewidth}{0.4pt}
\cfoot{\thepage}

% ---- Listas y símbolos
\usepackage{enumitem}
\setlist{itemsep=0.25em, topsep=0.3em}

% ---- Iconos y color
\usepackage{fontawesome5}
\usepackage{xcolor}
\usepackage[most]{tcolorbox}
\tcbuselibrary{breakable,skins}

% ---- Colores propios
\definecolor{uniPrimary}{HTML}{0A7AC3}
\definecolor{uniSoft}{HTML}{E9F4FB}
\definecolor{uniAccent}{HTML}{F5B700}
\definecolor{uniOK}{HTML}{2E7D32}
\definecolor{uniWarn}{HTML}{C62828}

% ---- Metadatos reutilizables
\newcommand{\asignatura}{Comercio Electrónico}
\newcommand{\tema}{TEMA 1 — Introducción al Comercio Electrónico}
\newcommand{\clase}{Clase 1}
\newcommand{\fecha}{\today}

% ---- Cabecera con metadatos
\lhead{\textsc{\asignatura}}
\rhead{\textbf{\tema}}

% ---- Estética de secciones
\usepackage{titlesec}
\titleformat{\section}{\Large\bfseries\color{uniPrimary}}{\thesection}{0.5em}{}
\titleformat{\subsection}{\large\bfseries}{\thesubsection}{0.5em}{}
\titleformat{\subsubsection}{\bfseries}{\thesubsubsection}{0.5em}{}

% ---- Cajas útiles
\newtcolorbox{ResumenBox}{
	title={\faHighlighter\; Resumen rápido},
	colback=uniSoft,
	colframe=uniPrimary,
	enhanced, breakable,
	left=8pt,right=8pt,top=8pt,bottom=8pt,
	boxrule=0.8pt,
	fonttitle=\bfseries
}

\newtcolorbox{DefBox}{
	title={\faBook\; Definición},
	colback=white,
	colframe=uniAccent,
	enhanced, breakable,
	left=8pt,right=8pt,top=8pt,bottom=8pt,
	boxrule=0.8pt,
	fonttitle=\bfseries
}

\newtcolorbox{NotaBox}{
	title={\faStickyNote\; Nota},
	colback=white,
	colframe=uniPrimary,
	enhanced, breakable,
	left=8pt,right=8pt,top=8pt,bottom=8pt,
	boxrule=0.8pt,
	fonttitle=\bfseries
}

\newtcolorbox{ChecklistBox}{
	title={\faTasks\; Ventajas / Inconvenientes},
	colback=white,
	colframe=uniPrimary,
	enhanced, breakable,
	left=8pt,right=8pt,top=8pt,bottom=8pt,
	boxrule=0.8pt,
	fonttitle=\bfseries
}

% =========================================================
\begin{document}

{\large \textbf{\asignatura} \;—\; \textbf{\tema} \hfill \textit{\clase, \fecha}}\\[0.4em]
\faUser\; Alumno/a: Alberto Díaz\hfill
\faChalkboardTeacher\; Vanesa Herrera Tirado\hfill

\vspace{0.6em}
\tableofcontents

% ---------------------------------------------------------
\section{¿Qué es el COE?}
\begin{DefBox}
El \textbf{COE} (Comercio Electrónico) es una disciplina que estudia cómo las empresas utilizan las redes digitales para realizar actividades comerciales, comunicarse con clientes y proveedores, y gestionar información empresarial.
\end{DefBox}

\begin{NotaBox}
El COE analiza el impacto de la tecnología en los hábitos de consumo, la comunicación, el marketing y los modelos de negocio.
\end{NotaBox}

% ---------------------------------------------------------
\section{¿Qué es el e-Commerce?}
\begin{DefBox}
Según Zwass (1996), el \textbf{comercio electrónico} es “compartir información empresarial, mantener relaciones comerciales y realizar transacciones comerciales a través de redes de telecomunicaciones”.
\end{DefBox}

\begin{ResumenBox}
El e-commerce consiste en comprar y vender bienes o servicios a través de Internet, utilizando medios electrónicos de pago y plataformas digitales.
\end{ResumenBox}

% ---------------------------------------------------------
\section{La revolución industrial y la Industria 4.0}
\subsection{Evolución histórica}
\begin{itemize}
	\item \textbf{1ª Revolución Industrial}: mecanización mediante la máquina de vapor.
	\item \textbf{2ª Revolución Industrial}: electricidad y producción en masa.
	\item \textbf{3ª Revolución Industrial}: automatización e Internet.
	\item \textbf{4ª Revolución Industrial (Industria 4.0)}: integración digital, IA, IoT y sistemas inteligentes.
\end{itemize}

\subsection{Concepto}
\begin{DefBox}
La \textbf{Industria 4.0} integra sistemas físicos y virtuales que cooperan a nivel global mediante tecnologías como el Internet de las Cosas, inteligencia artificial, robótica, big data y realidad aumentada.
\end{DefBox}

% ---------------------------------------------------------
\section{Sistemas empresariales digitales}

\subsection{EDI — Electronic Data Interchange}
\begin{DefBox}
El \textbf{EDI} (Intercambio Electrónico de Datos) permite que empresas intercambien información comercial (facturas, pedidos, inventarios) de forma automática y segura entre sus sistemas informáticos.
\end{DefBox}

\subsection{CRM — Customer Relationship Management}
\begin{DefBox}
El \textbf{CRM} es un sistema de gestión de relaciones con los clientes. Centraliza datos de ventas, marketing y soporte para fidelizar, personalizar ofertas y mejorar la atención al cliente.
\end{DefBox}

\subsection{Just in Time}
\begin{DefBox}
El método \textbf{Just in Time} busca producir y entregar solo lo necesario, en el momento exacto, reduciendo inventarios y desperdicio. Fue popularizado por Toyota.
\end{DefBox}

\subsection{ERP — Enterprise Resource Planning}
\begin{DefBox}
El \textbf{ERP} integra todas las áreas de una empresa (finanzas, recursos humanos, logística, ventas) en una única base de datos, mejorando la eficiencia y la trazabilidad de la información.
\end{DefBox}

\subsection{BI — Business Intelligence}
\begin{DefBox}
El \textbf{BI} (Inteligencia Empresarial) es el conjunto de herramientas y metodologías que permiten analizar datos para apoyar la toma de decisiones estratégicas.
\end{DefBox}

% ---------------------------------------------------------
\section{Tipos de modelos comerciales digitales (Diapositiva 95)}

\subsection{Perfiles comerciales}
\begin{itemize}
	\item \textbf{Productos propios}
	\item \textbf{Afiliación}
	\item \textbf{Dropshipping}
	\item \textbf{Suscripción}
	\item \textbf{Marketplaces}
\end{itemize}

\subsection{Tipos de relación}
\begin{itemize}
	\item \textbf{B2B (Business to Business)}: entre empresas.
	\item \textbf{B2C (Business to Consumer)}: empresa → consumidor.
	\item \textbf{C2C (Consumer to Consumer)}: entre particulares.
	\item \textbf{B2B2C / B2C2C}: modelos mixtos intermedios.
\end{itemize}

\subsection{Tipos de transacción}
\begin{itemize}
	\item Venta directa
	\item Venta flash
	\item Subastas
	\item Modelos mixtos
\end{itemize}

% ---------------------------------------------------------
\section{Ventajas e inconvenientes de la digitalización}

\begin{ChecklistBox}
\textbf{Ventajas:}
\begin{itemize}
	\item Mayor eficiencia y productividad.
	\item Acceso a mercados globales.
	\item Innovación y ventaja competitiva.
	\item Reducción de costes y tiempos.
	\item Adaptabilidad ante cambios.
\end{itemize}

\vspace{0.5em}
\textbf{Inconvenientes:}
\begin{itemize}
	\item Inversión inicial alta.
	\item Brecha digital y desigualdades tecnológicas.
	\item Complejidad de implementación.
	\item Riesgos de ciberseguridad.
	\item Impacto laboral (automatización).
\end{itemize}
\end{ChecklistBox}

% ---------------------------------------------------------
\section{Historia del e-Commerce}
\begin{itemize}
	\item \textbf{1960–70}: EDI y teleshopping.
	\item \textbf{1990s}: primeras webs comerciales (Pizza Hut, Lego, NetMarket).
	\item \textbf{2000–2020}: expansión global, apps móviles, redes sociales y logística avanzada.
	\item \textbf{2014–hoy}: auge del social commerce e integración omnicanal.
\end{itemize}

% ---------------------------------------------------------
\section{Presente y futuro del e-Commerce}

\subsection{Tendencias actuales}
\begin{itemize}
	\item \textbf{Social y Live Commerce}: compras en directo desde redes sociales.
	\item \textbf{M-Commerce}: crecimiento del comercio móvil.
	\item \textbf{Omnicanalidad}: integración entre tienda física y digital.
	\item \textbf{Pagos digitales}: wallets, Bizum, pagos fraccionados.
	\item \textbf{Logística avanzada}: entregas rápidas, sostenibilidad, trazabilidad.
\end{itemize}

\subsection{Tendencias futuras}
\begin{itemize}
	\item \textbf{Realidad virtual y aumentada} para experiencias inmersivas.
	\item \textbf{Asistentes virtuales y chatbots} para atención personalizada.
	\item \textbf{IA predictiva} para recomendaciones y previsión de demanda.
	\item \textbf{Blockchain y tokenización} en trazabilidad y pagos.
\end{itemize}

\end{document}
