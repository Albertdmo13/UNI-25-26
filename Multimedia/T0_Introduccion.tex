\documentclass[11pt,a4paper]{article}

% ---- Idioma y tipografía
\usepackage[spanish]{babel}
\usepackage[T1]{fontenc}
\usepackage[utf8]{inputenc}
\usepackage{lmodern}
\usepackage{microtype}

% ---- Márgenes y diseño
\usepackage[a4paper,margin=2.2cm]{geometry}
\usepackage{parskip}

% ---- Colores y cajas
\usepackage[most]{tcolorbox}
\usepackage{xcolor}
\usepackage{enumitem}
\usepackage{fontawesome5}

% ---- Colores propios
\definecolor{uniPrimary}{HTML}{0A7AC3}
\definecolor{uniSoft}{HTML}{E9F4FB}
\definecolor{uniAccent}{HTML}{F5B700}

% ---- Cajas útiles
\newtcolorbox{ObjetivosBox}{
	title={\faBullseye\; Objetivos},
	colback=uniSoft,
	colframe=uniPrimary,
	enhanced, breakable,
	left=8pt,right=8pt,top=8pt,bottom=8pt,
	boxrule=0.8pt,
	fonttitle=\bfseries
}

\newtcolorbox{NotaBox}{
	title={\faStickyNote\; Nota},
	colback=white,
	colframe=uniPrimary,
	enhanced, breakable,
	left=8pt,right=8pt,top=8pt,bottom=8pt,
	boxrule=0.8pt,
	fonttitle=\bfseries
}

\newtcolorbox{RecordatorioBox}{
	title={\faBell\; Recordatorio},
	colback=uniAccent!15,
	colframe=uniAccent,
	enhanced, breakable,
	left=8pt,right=8pt,top=8pt,bottom=8pt,
	boxrule=0.8pt,
	fonttitle=\bfseries
}

% ---- Documento
\begin{document}
	
	% Portada rápida
	\begin{center}
		{\huge \textbf{Apuntes de Multimedia}} \\[0.5cm]
		{\Large Universidad \;--\; Curso 2025} \\[0.3cm]
		\rule{0.8\linewidth}{0.5pt} \\[0.3cm]
		{\faUser\; Nombre del alumno: Alberto Díaz} \\[0.2cm]
		{\faChalkboardTeacher\; Profesor: Ana Fernandez} \\[1.2cm]
	\end{center}
	
	\section*{Introducción}
	
	\begin{ObjetivosBox}
		\begin{itemize}[leftmargin=1.5em]
			\item Identificar contenidos digitales.
			\item Conocer técnicas de comprensión multimedia.
		\end{itemize}
	\end{ObjetivosBox}
	
	\subsection*{Evaluación}
	
	\begin{center}
		\renewcommand{\arraystretch}{1.4} % más espacio entre filas
		\begin{tabular}{|l|c|c|c|}
			\hline
			\textbf{Criterio} & \textbf{Porcentaje} & \textbf{Obligatorio} & \textbf{Recuperable} \\
			\hline
			Teoría (25\% cada parcial) & 50\% & Sí & Sí \\
			Práctica & 25\% & Sí & Sí \\
			Aprovechamiento en clase & 10\% & No & No \\
			Trabajo teórico & 15\% & No & Sí \\
			\hline
			\textbf{Total} & \textbf{100\%} & & \\
			\hline
		\end{tabular}
	\end{center}
	
	\begin{NotaBox}
		\textbf{Requisito:} Hay que sacar al menos un \textbf{5 en cada parcial} para que haga media.
	\end{NotaBox}
	
	\begin{RecordatorioBox}
		\begin{itemize}
			\item La \textbf{parte práctica} consiste en desarrollar una aplicación (probablemente en grupos).  
			\textit{Inicio: 6 de Octubre. Presentación y defensa incluidas.}
			\item La \textbf{participación en clase} se evalúa contestando preguntas, atendiendo y siendo activo.
		\end{itemize}
	\end{RecordatorioBox}
	
\end{document}
