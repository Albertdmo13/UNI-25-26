\documentclass[11pt,a4paper]{article}

\usepackage{graphicx}

% ---- Idioma y tipografía
\usepackage[spanish]{babel}
\usepackage[T1]{fontenc}
\usepackage[utf8]{inputenc}
\usepackage{lmodern}
\usepackage{microtype}

% ---- Márgenes y diseño
\usepackage[a4paper,margin=2.2cm]{geometry}
\usepackage{parskip}

% ---- Enlaces
\usepackage[hidelinks]{hyperref}
\usepackage{bookmark}

% ---- Cabeceras y pies
\usepackage{fancyhdr}
\pagestyle{fancy}
\fancyhf{}
\renewcommand{\headrulewidth}{0.4pt}
\cfoot{\thepage}

% ---- Listas y símbolos
\usepackage{enumitem}
\setlist{itemsep=0.25em, topsep=0.3em}

% ---- Iconos y color
\usepackage{fontawesome5}
\usepackage{xcolor}
\usepackage[most]{tcolorbox}
\tcbuselibrary{breakable,skins}

% ---- Colores propios
\definecolor{uniPrimary}{HTML}{0A7AC3}
\definecolor{uniSoft}{HTML}{E9F4FB}
\definecolor{uniAccent}{HTML}{F5B700}
\definecolor{uniOK}{HTML}{2E7D32}
\definecolor{uniWarn}{HTML}{C62828}

% ---- Metadatos reutilizables
\newcommand{\asignatura}{MULTIMEDIA}
\newcommand{\tema}{TEMA 3 — Algoritmos de compresión}
\newcommand{\clase}{Clase 3}
\newcommand{\fecha}{\today}

\usepackage{booktabs}
\usepackage{siunitx}
\usepackage{amsmath}

% ---- Cabecera con metadatos
\lhead{\textsc{\asignatura}}
\rhead{\textbf{\tema}}

% ---- Estética de secciones
\usepackage{titlesec}
\titleformat{\section}{\Large\bfseries\color{uniPrimary}}{\thesection}{0.5em}{}
\titleformat{\subsection}{\large\bfseries}{\thesubsection}{0.5em}{}
\titleformat{\subsubsection}{\bfseries}{\thesubsubsection}{0.5em}{}

% ---- Cajas útiles
\newtcolorbox{ObjetivosBox}{
	title={\faBullseye\; Índice},
	colback=uniSoft,
	colframe=uniPrimary,
	enhanced, breakable,
	left=8pt,right=8pt,top=8pt,bottom=8pt,
	boxrule=0.8pt,
	fonttitle=\bfseries
}

\newtcolorbox{DefBox}{
	title={\faBook\; Definición},
	colback=white,
	colframe=uniAccent,
	enhanced, breakable,
	left=8pt,right=8pt,top=8pt,bottom=8pt,
	boxrule=0.8pt,
	fonttitle=\bfseries
}

\newtcolorbox{NotaBox}{
	title={\faStickyNote\; Nota},
	colback=white,
	colframe=uniPrimary,
	enhanced, breakable,
	left=8pt,right=8pt,top=8pt,bottom=8pt,
	boxrule=0.8pt,
	fonttitle=\bfseries
}

\newtcolorbox{RecordatorioBox}{
	title={\faBell\; Recordatorio},
	colback=uniAccent!15,
	colframe=uniAccent,
	enhanced, breakable,
	left=8pt,right=8pt,top=8pt,bottom=8pt,
	boxrule=0.8pt,
	fonttitle=\bfseries
}

\newtcolorbox{ChecklistBox}{
	title={\faTasks\; Tareas / Checklist},
	colback=white,
	colframe=uniPrimary,
	enhanced, breakable,
	left=8pt,right=8pt,top=8pt,bottom=8pt,
	boxrule=0.8pt,
	fonttitle=\bfseries
}

\newtcolorbox{ResumenBox}{
	title={\faHighlighter\; Resumen rápido (5 líneas)},
	colback=uniSoft,
	colframe=uniPrimary,
	enhanced, breakable,
	left=8pt,right=8pt,top=8pt,bottom=8pt,
	boxrule=0.8pt,
	fonttitle=\bfseries
}

\newtcolorbox{VocabBox}{
	title={\faLanguage\; Vocabulario clave},
	colback=white,
	colframe=uniPrimary,
	enhanced, breakable,
	left=8pt,right=8pt,top=8pt,bottom=8pt,
	boxrule=0.8pt,
	fonttitle=\bfseries
}

% =========================================================
\begin{document}

\section{Algoritmos con pérdidas}

\subsection{Codificación Aritmética}

A cada símbolo se le asigna un intervalo en el rango [0, 1) proporcional a su probabilidad. A medida que se procesan los símbolos, el intervalo se reduce según el símbolo actual, lo que permite representar la secuencia completa con un solo número dentro del intervalo final.

Se basa en una función de distribución.

\textbf{La función de distribución} es la función que especifica la probabilidad de que una variable aleatoria sea menor o igual que un valor dado.

\[F_X(x) = P(X \leq x)\]

\textbf{La función de probabilidad} es una función que describe la probabilidad de que una variable aleatoria continua tome un valor específico.

\[ f_X(x) = \frac{d}{dx} F_X(x) \]

En el caso de una variable aleatoria continua, la función toma valores en un rango continuo.

\textbf{La función de distribución cumple}

\[ \lim_{x \to \infty} F_X(x) = 1 \]

Es monotona y creciente.

La función de probabilidad es la derivada de la función de distribución.

\[ F_X(x) = \int_{-\infty}^{x} f_X(t) \,dt \]

Si tenemos un alfabeto ordenado \( S = \{s_1, s_2, \ldots, s_n\} \) con probabilidades \( P(s_i) \), se define la función de distribución de la probabilidad acumulada \( C(s_i) \) como:
\[ C(s_i) = \sum_{j=1}^{i} P(s_j) \]

Para codificar una secuencia de símbolos \( s_1, s_2, \ldots, s_k \), se sigue el siguiente procedimiento:
\begin{enumerate}
	\item Inicializar el intervalo \([low, high) = [0, 1)\).
	\item Para cada símbolo \( s_i \) en la secuencia:
	\begin{itemize}
		\item Calcular el rango actual: \( range = high - low \).
		\item Actualizar los límites del intervalo:
		\[
		high = low + range \times C(s_i)
		\]
		\[
		low = low + range \times C(s_{i-1})
		\]
	\end{itemize}
	\item Al finalizar, cualquier número dentro del intervalo final \([low, high)\) puede representar la secuencia codificada.

\end{enumerate}

\subsection{Decodificación aritmética}

Algoritmo de Decodificación recursiva de una etiqueta E con n caracteres dada una fuente F(A, P)

\begin{itemize}
	\item Calcular la distribución de probabilidad
	\item Inicializar $low = 0$, $high = 1$
	\item Para i desde 1 hasta n hacer:
\[V = \frac{E - low}{high - low}\]
	\item Buscar el símbolo $s$ tal que $C(s-1) \leq V < C(s)$
	\item Actualización de los límites:
\[high = low + (high - low) * C(s)\]
\[low = low + (high - low) * C(s-1)\]
\end{itemize}

\end{document}
