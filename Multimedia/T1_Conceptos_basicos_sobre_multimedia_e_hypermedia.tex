\documentclass[11pt,a4paper]{article}

% ---- Idioma y tipografía
\usepackage[spanish]{babel}
\usepackage[T1]{fontenc}
\usepackage[utf8]{inputenc}  % (si usas XeLaTeX/LuaLaTeX, elimínala)
\usepackage{lmodern}
\usepackage{microtype}

% ---- Márgenes y diseño
\usepackage[a4paper,margin=2.2cm]{geometry}
\usepackage{parskip}

% ---- Enlaces
\usepackage[hidelinks]{hyperref}
\usepackage{bookmark}
% ---- Cabeceras y pies
\usepackage{fancyhdr}
\pagestyle{fancy}
\fancyhf{}
\renewcommand{\headrulewidth}{0.4pt}
\cfoot{\thepage}

% ---- Listas y símbolos
\usepackage{enumitem}
\setlist{itemsep=0.25em, topsep=0.3em}

% ---- Iconos y color
\usepackage{fontawesome5}
\usepackage{xcolor}
\usepackage[most]{tcolorbox}
\tcbuselibrary{breakable,skins}

% ---- Colores propios
\definecolor{uniPrimary}{HTML}{0A7AC3}
\definecolor{uniSoft}{HTML}{E9F4FB}
\definecolor{uniAccent}{HTML}{F5B700}
\definecolor{uniOK}{HTML}{2E7D32}
\definecolor{uniWarn}{HTML}{C62828}

% ---- Metadatos reutilizables
\newcommand{\asignatura}{MULTIMEDIA}
\newcommand{\tema}{TEMA 1 — Conceptos básicos sobre multimedia e Hipermedia}  % <-- cambia el título del tema aquí
\newcommand{\clase}{Clase 1}
\newcommand{\fecha}{\today}                              % <-- pon la fecha de la clase

% ---- Cabecera con metadatos
\lhead{\textsc{\asignatura}}
\chead{\textbf{\tema}}
\rhead{\clase\;|\; \fecha}

% ---- Estética de secciones
\usepackage{titlesec}
\titleformat{\section}{\Large\bfseries\color{uniPrimary}}{\thesection}{0.5em}{}
\titleformat{\subsection}{\large\bfseries}{\thesubsection}{0.5em}{}
\titleformat{\subsubsection}{\bfseries}{\thesubsubsection}{0.5em}{}

% ---- Cajas útiles
\newtcolorbox{ObjetivosBox}{
	title={\faBullseye\; Índice},
	colback=uniSoft,
	colframe=uniPrimary,
	enhanced, breakable,
	left=8pt,right=8pt,top=8pt,bottom=8pt,
	boxrule=0.8pt,
	fonttitle=\bfseries
}

\newtcolorbox{DefBox}{
	title={\faBook\; Definición},
	colback=white,
	colframe=uniAccent,
	enhanced, breakable,
	left=8pt,right=8pt,top=8pt,bottom=8pt,
	boxrule=0.8pt,
	fonttitle=\bfseries
}

\newtcolorbox{NotaBox}{
	title={\faStickyNote\; Nota},
	colback=white,
	colframe=uniPrimary,
	enhanced, breakable,
	left=8pt,right=8pt,top=8pt,bottom=8pt,
	boxrule=0.8pt,
	fonttitle=\bfseries
}

\newtcolorbox{RecordatorioBox}{
	title={\faBell\; Recordatorio},
	colback=uniAccent!15,
	colframe=uniAccent,
	enhanced, breakable,
	left=8pt,right=8pt,top=8pt,bottom=8pt,
	boxrule=0.8pt,
	fonttitle=\bfseries
}

\newtcolorbox{ChecklistBox}{
	title={\faTasks\; Tareas / Checklist},
	colback=white,
	colframe=uniPrimary,
	enhanced, breakable,
	left=8pt,right=8pt,top=8pt,bottom=8pt,
	boxrule=0.8pt,
	fonttitle=\bfseries
}

\newtcolorbox{ResumenBox}{
	title={\faHighlighter\; Resumen rápido (5 líneas)},
	colback=uniSoft,
	colframe=uniPrimary,
	enhanced, breakable,
	left=8pt,right=8pt,top=8pt,bottom=8pt,
	boxrule=0.8pt,
	fonttitle=\bfseries
}

\newtcolorbox{VocabBox}{
	title={\faLanguage\; Vocabulario clave},
	colback=white,
	colframe=uniPrimary,
	enhanced, breakable,
	left=8pt,right=8pt,top=8pt,bottom=8pt,
	boxrule=0.8pt,
	fonttitle=\bfseries
}

% ---- Tabla estilo evaluación (por si la necesitas hoy)
\newcommand{\TablaEvaluacion}{
	\begin{center}
		\renewcommand{\arraystretch}{1.3}
		\begin{tabular}{|l|c|c|c|}
			\hline
			\textbf{Criterio} & \textbf{Porcentaje} & \textbf{Obligatorio} & \textbf{Recuperable} \\
			\hline
			Teoría (25\% cada parcial) & 50\% & Sí & Sí \\
			Práctica & 25\% & Sí & Sí \\
			Aprovechamiento en clase & 10\% & No & No \\
			Trabajo teórico & 15\% & No & Sí \\
			\hline
			\textbf{Total} & \textbf{100\%} &  &  \\
			\hline
		\end{tabular}
	\end{center}
}

% =========================================================
\begin{document}

	% ---- Cabecera de ficha de clase (rellenable en cada sesión)
	{\large \textbf{\asignatura} \;—\; \textbf{\tema} \hfill \textit{\clase, \fecha}}\\[0.4em]
	\faUser\; Alumno/a: Alberto Díaz\hfill
	\faChalkboardTeacher\; Profesor/a: Ana Fernández

	\vspace{0.6em}

	\tableofcontents

	% ---- Conceptos clave y definiciones
	\section{Definiciones}
	\begin{DefBox}
		\textbf{Multimedia:} Representación integrada de la información en forma de texto, gráficos, imágenes, vídeo.

		La multimedia no tiene porque ser interactiva. No necesariamente tiene que ser digital.
	\end{DefBox}



	\begin{DefBox}
		\textbf{Aplicación multimedia:} Software capaz de ofrecer información al usuario integrando distintos medios de representación multimedia.

		La información multimedia se procesa y almacena digitalmente.
	\end{DefBox}

	\begin{DefBox}
		\textbf{Elemento Interactivo: } Un elemento es interactivo cuando responde a la entrada de información del usuario. El usuario debe poder tomar una decisión de iniciar la interacción, de forma explicita o implícita.
	\end{DefBox}

	\begin{DefBox}
		\textbf{Hipertexto} Sistema de organización de datos basado en la vinculación de bloques de información llamados nodos.
	\end{DefBox}

	\begin{DefBox}
		\textbf{Hipermedia} Hipertexto + Multimedia. Conectar fragmentos de información que no necesariamente tienen porque ser texto si no cualquier tipo de multimedia.
	\end{DefBox}

	\subsection{Diferencias entre Multimedia, Hipertexto y Hipermedia}

	\begin{center}
		\renewcommand{\arraystretch}{1.4} % más espacio entre filas
		\begin{tabular}{|l|c|c|c|}
			\hline
			\textbf{Criterio} & \textbf{Multimedia} & \textbf{Hipertexto} & \textbf{Hipermedia} \\
			\hline
			\textbf{Contenido}
			& Más de un medio
			& Solo un medio (texto)
			& Más de un medio (texto + otros) \\
			\hline
			\textbf{Enlaces}
			& No
			& Sí (entre textos)
			& Sí (entre distintos medios) \\
			\hline
			\textbf{Navegación}
			& No (Linear)
			& Sí (No linear)
			& Sí \\
			\hline
		\end{tabular}
	\end{center}

	\subsection{Medios multimedia y Contexto}
	\begin{DefBox}
		\textbf{Medio} Canal que permite la distribución y comunicación de la información: Texto, gráficos, imagen, audio y vídeo.
	\end{DefBox}

	\subsubsection{Clasificación en función del tiempo}

	\begin{itemize}[leftmargin=1.5em]
		\item \textbf{Medios continuos (dependientes del tiempo):}
		Requieren una secuencia temporal para ser comprendidos.
		Ejemplos: \textit{audio, vídeo}.

		\item \textbf{Medios discretos (independientes del tiempo):}
		No dependen de la dimensión temporal para interpretarse.
		Ejemplos: \textit{texto, imágenes}.
	\end{itemize}

	\begin{NotaBox}
		\textbf{Contexto:} Esta clasificación se aplica en el ámbito de las \textbf{comunicaciones audiovisuales digitales}.
	\end{NotaBox}

	\subsection*{La importancia del usuario}
	\begin{DefBox}
		\textbf{Percepción} La forma en la que el cerebro interpreta los estímulos del exterior. Las percepciones evocan \textbf{sensaciones}.
	\end{DefBox}

	\begin{DefBox}
		\textbf{Sensaciones} Impresiones que los estímulos externos generan sobre las personas. Percepción de un cambio o desequilibrio.
	\end{DefBox}

	\begin{DefBox}
		\textbf{Emociones} Respuesta que aparece después de la sensación, alegría, tristeza...
	\end{DefBox}

	\textbf{Ejemplo}:
	Veo una nube negra $\rightarrow$ \textit{percepción}.
	Pienso que va a llover $\rightarrow$ \textit{sensación}.
	Me invade la tristeza $\rightarrow$ \textit{emoción}.

	\section{Reseña Histórica}
	Primeras comunicaciones visuales surgían en la prehistoria pero no eran multimedia

	El invento del \textbf{transistor} fue importante para el desarrollo de la multimedia.

	\begin{itemize}[]
		\item \textbf{1945} Vannebar Bush propuso que los computadores deberían usarse como soporte del trabajo intelectual. Almacenamiento y comunicación de contenido multimedia y conocimiento de la humanidad. \textbf{(MEMEX) Memory Extension.}

		\item \textbf{1983} Se desarrolla \textbf{Intermedia}. Programación de creación hipertextual para UNIX.

		\item \textbf{1984} Apple lanza el \textbf{primer Macintosh} con interfaz gráfica. Primera computadora con altas capacidades de reproducción de sonidos y diseño gráfico.

		\item \textbf{1980´s} Aparecen \textbf{videojuegos} y software de entretenimiento.

		\item \textbf{1992} Es posible integrar audio, sonido  y voz, gráficas, animación de texto. Se expande la world wide web.

		\item \textbf{1995-2016} Gran evolución. Aparece el \textbf{metaverso}.
	\end{itemize}

	\subsection{Punto de inflexión: 1995-2000}

	\begin{itemize}[]
		\item Se extiende la señal analógica TV.
		\item \textbf{VHS}: Principal medio para grabar.
		\item Los teléfonos solo tenían llamadas y SMS.
		\item Conexión a internet es lenta. Modem.
		\item Páginas web son \textbf{estáticas}
	\end{itemize}

	\subsection{Punto de inflexión: 2015-2016}

	\begin{itemize}[]
		\item Señal digital de TV
		\item Teléfonos móviles son \textbf{smartphones} con muchas funciones.
		\item Páginas web son sitios interactivos, dinámicas.
		\item Mejor conexión a internet.
		\item Videollamadas.
		\item Se desarrollan las redes sociales.
	\end{itemize}

	\subsection{Punto de inflexión: Escena actual}

	\begin{itemize}[]
		\item Inteligencia Artificial.
		\item Segmentación de imágenes en tiempo real
		\item Realidad Virtual y Aumentada
		\item ChatGPT
	\end{itemize}

\section{Servicios multimedia}

\begin{DefBox}
\textbf{Servicio multimedia:} Permite manejar desde un terminal todas las formas de información electrónica conocidas (texto, gráficos, audio, vídeo, fotografías, música y comunicaciones telefónicas).
Una de sus características principales es la \textbf{interactividad}, que posibilita la comunicación con otras personas o dispositivos.
\end{DefBox}

\subsection{Clasificación de los servicios multimedia}
\begin{itemize}[leftmargin=1.5em]
  \item \textbf{Servicios de broadcasting convencional}
  \item \textbf{Servicios de broadcasting interactivos}
  \item \textbf{Servicios de reproducción multimedia}
  \item \textbf{Servicios de comunicaciones personales}
  \item \textbf{Servicios de juegos}
  \item \textbf{Servicios de monocasting compartido}
\end{itemize}

\subsection{Características principales por tipo}

\subsubsection*{Broadcasting convencional}
\begin{itemize}
  \item Unidireccional
  \item Punto a multipunto
  \item Tiempo real o no
  \item Bajo retraso (no crítico)
  \item Alta calidad
  \item Modelo de producción de contenidos centralizados
  \item Soporte de varios canales y redes
\end{itemize}

\subsubsection*{Broadcasting interactivo}
\begin{itemize}
  \item Bidireccional (asimétrico)
  \item Punto a multipunto y punto a punto
  \item Tiempo real o no
  \item Tiempo de respuesta crítico
  \item Alta calidad
  \item Producción centralizada de contenidos
  \item Uso de múltiples canales y redes
\end{itemize}

\subsubsection*{Reproducción multimedia}
\begin{itemize}
  \item Local (no hay transmisión en red)
  \item Alta capacidad de almacenamiento (p. ej. discos ópticos)
  \item Bajo retraso
  \item Calidad muy alta
\end{itemize}

\subsubsection*{Comunicaciones personales}
Discord, Teams, WhatsApp
\begin{itemize}
  \item Bidireccionales y simétricas
  \item Punto a punto
  \item Tiempo real
  \item Retraso crítico
  \item Calidad media/baja
  \item Contenidos específicos
  \item Soporte de varios canales/redes
\end{itemize}

\subsubsection*{Servicios de juegos}
\begin{itemize}
  \item Bidireccionales (almacenamiento y transmisión)
  \item Punto a punto o multipunto
  \item Tiempo real
  \item Retraso crítico
  \item Alta calidad y realismo
  \item Contenidos sintéticos y reales
  \item Uso de múltiples canales y redes
\end{itemize}

\subsubsection*{Monocasting compartido}
\begin{itemize}
  \item Bidireccional (asimétrico)
  \item Descarga/visionado
  \item Punto a punto
  \item Tiempo real (descarga) y no real (subida)
  \item Retraso crítico
  \item Amplio rango de calidades
  \item Modelo de producción descentralizado y compartido
  \item Uso de múltiples canales/redes sociales
\end{itemize}

El contenido descentralizado se refiere a aquel que es generado y distribuido por múltiples usuarios o entidades independientes, en lugar de depender de una única fuente central. Por ejemplo, plataformas como Netflix gestionan y controlan todo su catálogo (contenido centralizado), mientras que en YouTube cualquier usuario puede crear y compartir vídeos, lo que constituye un modelo de contenido descentralizado.

\section{Señales analógicas vs digitales}

\begin{DefBox}
\textbf{Señales analógicas:} Representan fenómenos continuos de la naturaleza (voz, luz, temperatura, etc.). Son percibidas directamente por el ser humano a través de los sentidos.
Se caracterizan por ser infinitas en valores y difíciles de controlar totalmente.
\end{DefBox}

\begin{DefBox}
\textbf{Señales digitales:} Representación discreta de una señal analógica mediante procesos de \textbf{muestreo}, \textbf{cuantificación} y \textbf{codificación}.
Permiten mayor control, almacenamiento y transmisión eficiente.
\end{DefBox}

\subsection{Características principales}

\subsubsection*{Señales analógicas}
\begin{itemize}
  \item Continuas en el tiempo y en amplitud.
  \item Utilizadas en sistemas tradicionales de comunicación (TV, radio, teléfono).
  \item Menor control sobre la calidad (ruido, distorsión, interferencias).
  \item Requieren equipos especializados para almacenamiento y transmisión.
\end{itemize}

\subsubsection*{Señales digitales}
\begin{itemize}
  \item Discretas en el tiempo y en amplitud (valores finitos representados en bits).
  \item Flexibles gracias al procesado por software.
  \item Almacenamiento sencillo y de gran capacidad.
  \item Permiten compresión y transmisión eficiente.
  \item Menor degradación en transmisión y copias.
\end{itemize}

\subsection{Ventajas de lo digital frente a lo analógico}
\begin{ChecklistBox}
\begin{itemize}
  \item Mayor \textbf{flexibilidad}: procesamiento mediante software frente al hardware analógico.
  \item Implementación de algoritmos avanzados de procesado de señal.
  \item Reducción de costes a largo plazo.
  \item Facilidad de almacenamiento masivo.
  \item Menor sensibilidad al ruido e interferencias.
\end{itemize}
\end{ChecklistBox}

\subsection{Proceso de conversión}
\begin{itemize}
  \item \textbf{Muestreo:} Se toma la señal analógica a intervalos regulares de tiempo.
  \item \textbf{Cuantificación:} Se aproxima el valor muestreado a un nivel discreto predeterminado.
  \item \textbf{Codificación:} Representación en bits de cada valor cuantificado.
\end{itemize}

\begin{NotaBox}
Si el emisor y el receptor son humanos, se requiere \textbf{ADC (Conversor Analógico-Digital)} en la entrada y \textbf{DAC (Conversor Digital-Analógico)} en la salida.
Si ambos son dispositivos electrónicos, la transmisión puede permanecer en el dominio digital.
\end{NotaBox}

\begin{ResumenBox}
Las señales analógicas son naturales y continuas, pero difíciles de controlar.
Las digitales permiten un manejo más eficiente y robusto, gracias a la conversión mediante muestreo, cuantificación y codificación.
Su ventaja radica en la flexibilidad, el almacenamiento y la resistencia al ruido.
\end{ResumenBox}

\subsection{Teorema de Nyquist}

\begin{DefBox}
\textbf{Teorema de Nyquist:}
Para poder reconstruir una señal analógica a partir de sus muestras digitales, la \textbf{frecuencia de muestreo} $F_m$ debe ser al menos el doble del \textbf{ancho de banda} $B$ de la señal original:

\[
F_m > 2B
\]

Esto garantiza una representación digital adecuada y evita pérdidas de información.
\end{DefBox}

\subsubsection*{Conceptos clave}
\begin{itemize}
  \item \textbf{Ancho de banda ($B$):} Rango de frecuencias presentes en la señal.
  \item \textbf{Frecuencia de muestreo ($F_m$):} Número de muestras tomadas por segundo.
  \item \textbf{Aliasing:} Distorsión que ocurre si se muestrea a una frecuencia inferior a $2B$.
  Algunas frecuencias de la señal original se confunden con otras diferentes al reconstruir la señal.
\end{itemize}

\subsubsection*{Ejemplo práctico}
\begin{itemize}
  \item La voz humana tiene un ancho de banda aproximado de $3\,\text{kHz}$.
  \item Según Nyquist: $F_m > 2 \times 3\,\text{kHz} = 6\,\text{kHz}$.
  \item En telefonía digital se usa una frecuencia de muestreo estándar de $8\,\text{kHz}$ para asegurar una buena calidad.
\end{itemize}

\begin{NotaBox}
Cuanto mayor es la frecuencia de muestreo, más fiel será la representación de la señal, pero también mayor será el \textbf{bit rate} y el almacenamiento necesario.
Por ello, se busca un compromiso entre calidad y eficiencia.
\end{NotaBox}

\begin{ResumenBox}
El teorema de Nyquist establece que debemos muestrear al doble de la frecuencia máxima de la señal para evitar aliasing.
Es la base de toda digitalización de audio, vídeo y otros datos multimedia.
\end{ResumenBox}

\subsection{Cuantificación}

\begin{DefBox}
\textbf{Cuantificación:}
Proceso mediante el cual, tras el muestreo de una señal analógica, cada valor continuo se aproxima al \textbf{nivel discreto más cercano}.
De esta forma, los valores de la señal se representan con un número finito de niveles llamados \textbf{niveles de cuantificación}.
\end{DefBox}

\subsubsection*{Conceptos clave}
\begin{itemize}
  \item La cuantificación convierte una secuencia de muestras continuas en un conjunto finito de valores discretos.
  \item El número de niveles de cuantificación $N$ depende del número de bits $b$ utilizados:
  \[
  N = 2^b
  \]
  \item Cada muestra tiene un \textbf{error de cuantificación}:
  \[
  e(T) = x_{\text{analógica}}(T) - x_{\text{cuantificada}}(T)
  \]
  Cuantos más bits se usan, menor es el error, pero mayor el espacio necesario para almacenamiento o transmisión.
\end{itemize}

\subsubsection*{Tipos de cuantificación}
\begin{itemize}
  \item \textbf{Uniforme:} Los niveles están equiespaciados.
  Se utiliza en imágenes o datos en general.
  \item \textbf{No uniforme:} Los niveles no están equiespaciados; se adaptan a la señal.
  Muy útil en audio, ya que hay muchas amplitudes cercanas a cero (silencios).
  \item \textbf{Midtread:} Tiene un nivel de cuantificación en cero.
  Adecuado para señales con muchas muestras alrededor del valor cero (como audio).
  \item \textbf{Midrise:} No tiene nivel en cero, el número de niveles es par.
\end{itemize}

\subsubsection*{Ejemplo}
\begin{itemize}
  \item Para un cuantificador de 2 bits ($b=2$):
  \[
  N = 2^2 = 4 \text{ niveles discretos}
  \]
  La señal se aproxima solo a 4 valores posibles, lo que genera un error de cuantificación alto.
  \item Si se usan 8 bits, se obtienen $2^8 = 256$ niveles, con una representación mucho más fiel.
\end{itemize}

\begin{NotaBox}
La \textbf{calidad de la cuantificación} se mide con la relación señal-ruido (SNR).
Se cumple que, aproximadamente, la SNR mejora \textbf{6 dB por cada bit adicional} en el cuantificador.
\end{NotaBox}

\begin{ResumenBox}
La cuantificación aproxima cada muestra a un nivel discreto predeterminado.
A mayor número de bits, menor error y mayor calidad, pero también mayor consumo de memoria y ancho de banda.
\end{ResumenBox}

% ------------------------------------------------------------
% Sección ampliada: cuantificación, bitrate y SNR
% ------------------------------------------------------------
\subsection{Cuantificación: bitrate y SNR}

La \emph{cuantificación} asigna cada muestra analógica a uno de los $2^N$ niveles
posibles, donde $N$ es el número de bits por muestra. En audio PCM (modulación por
impulso codificado), el \textbf{bitrate} (tasa de bits) viene dado por:
\begin{equation}
  R \;=\; f_s \cdot N \cdot C \quad [\text{bits/s}]
\end{equation}
donde $f_s$ es la frecuencia de muestreo (muestras por segundo),
$N$ los bits por muestra, y $C$ el número de canales (por ejemplo, $C=2$ en estéreo).
Para pasar a bytes por segundo, dividir entre 8.

\paragraph{Ejemplos de bitrate PCM sin compresión}
\begin{itemize}
  \item \textbf{CD audio estéreo}: $f_s=44{,}1$ kHz, $N=16$, $C=2$ \\
    $R = 44{,}100 \times 16 \times 2 = 1{,}411{,}200\ \text{bits/s} \approx 1{,}411\ \text{kbit/s}$ \\
    $= 176{,}400\ \text{B/s} \approx 172{,}27\ \text{KiB/s} \approx 10{,}09\ \text{MiB/min}$.
  \item \textbf{44{,}1 kHz, 16 bits, mono} ($C=1$):\\
    $R = 44{,}100 \times 16 \times 1 = 705{,}600\ \text{bits/s} \approx 706\ \text{kbit/s}$ \\
    $\approx 5{,}05\ \text{MiB/min}$.
  \item \textbf{Estándar de producción} $48$ kHz, $24$ bits, estéreo ($C=2$):\\
    $R = 48{,}000 \times 24 \times 2 = 2{,}304{,}000\ \text{bits/s} \approx 2{,}304\ \text{kbit/s}$ \\
    $= 288{,}000\ \text{B/s} \approx 16{,}48\ \text{MiB/min}$.
  \item \textbf{Alta resolución} $96$ kHz, $24$ bits, estéreo: \\
    $R = 96{,}000 \times 24 \times 2 = 4{,}608{,}000\ \text{bits/s} \approx 4{,}608\ \text{kbit/s}$ \\
    $= 576{,}000\ \text{B/s} \approx 32{,}96\ \text{MiB/min}$.
\end{itemize}

\paragraph{Bitrates típicos en audio con compresión (referencia)}
Estos valores no siguen la fórmula anterior porque usan codificación perceptual (con o sin pérdidas):
\begin{itemize}
  \item \textbf{MP3}: 128, 192, 256, 320 kbit/s (estéreo).
  \item \textbf{AAC}: 128--256 kbit/s (estéreo) para buena calidad; en streaming es común 256 kbit/s.
  \item \textbf{Opus}: 64--160 kbit/s (estéreo) según calidad/latencia; voz mono a menudo 16--32 kbit/s.
\end{itemize}

\paragraph{Cuantización y ruido: SNR teórico}
La cuantificación uniforme introduce un error de redondeo que se modela como ruido blanco acotado.
Para una \textbf{señal senoidal a escala completa} (full-scale), el \textbf{SNR de cuantificación} ideal es:
\begin{equation}
  \mathrm{SNR}_q \;\approx\; 6{,}02\,N + 1{,}76 \quad [\text{dB}].
\end{equation}
De esta relación se desprende que cada bit adicional aporta $\approx 6$ dB de mejora en SNR.

\paragraph{Ejemplos de SNR teórico}
\begin{itemize}
  \item $N=8$ bits $\Rightarrow \mathrm{SNR}_q \approx 6{,}02\times 8 + 1{,}76 \approx 49{,}9\ \text{dB}$.
  \item $N=16$ bits $\Rightarrow \mathrm{SNR}_q \approx 98{,}1\ \text{dB}$ (típico de CD en el mejor caso).
  \item $N=24$ bits $\Rightarrow \mathrm{SNR}_q \approx 146{,}2\ \text{dB}$ (límite teórico; en la práctica
        el ruido analógico del convertidor/previos reduce esta cifra).
\end{itemize}

\paragraph{ENOB (número efectivo de bits)}
En sistemas reales, el SNR medido puede traducirse a \emph{bits efectivos}:
\begin{equation}
  \mathrm{ENOB} \;=\; \frac{\mathrm{SNR}_{\text{medido}} - 1{,}76}{6{,}02}.
\end{equation}
\emph{Ejemplo}: si se mide $\mathrm{SNR}=90$ dB, entonces $\mathrm{ENOB}\approx (90-1{,}76)/6{,}02 \approx 14{,}7$ bits.

\paragraph{Notas prácticas}
\begin{itemize}
  \item Subir $f_s$ aumenta el \emph{bitrate} linealmente; subir $N$ mejora el \emph{SNR} teórico $\approx 6$ dB/bit.
  \item \emph{Dither} y \emph{noise shaping} pueden mejorar artefactos perceptuales al reducir a menor $N$.
  \item En compresión con pérdidas (p.\,ej., MP3/AAC/Opus) el \emph{bitrate} ya no depende de $f_s$, $N$ y $C$ de forma directa:
        el códec descarta información psicoacústicamente redundante para alcanzar el objetivo en kbit/s.
\end{itemize}

% (Opcional) Tabla rápida de bitrates PCM comunes.
\begin{table}[h!]
\centering
\begin{tabular}{lccc}
\hline
\textbf{Formato} & $f_s$ (kHz) & $N$ (bits) & \textbf{Bitrate} (kbit/s) \\
\hline
Mono ``CD-like''     & 44{,}1 & 16 & 706 \\
Estéreo CD           & 44{,}1 & 16 & 1{,}411 \\
Estéreo producción   & 48     & 24 & 2{,}304 \\
Estéreo alta resolución & 96  & 24 & 4{,}608 \\
\hline
\end{tabular}
\caption{Bitrates PCM sin compresión típicos (aprox.).}
\end{table}


\section{Resumen rápido}

\begin{ResumenBox}
Los servicios multimedia pueden ser unidireccionales o bidireccionales, en tiempo real o bajo demanda. Su calidad, retraso y modelo de producción varían según el tipo de servicio: desde la difusión masiva centralizada (broadcasting) hasta la compartición descentralizada en redes sociales (monocasting compartido).
\end{ResumenBox}

\end{document}
