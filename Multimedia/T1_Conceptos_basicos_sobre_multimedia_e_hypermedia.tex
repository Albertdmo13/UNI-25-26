\documentclass[11pt,a4paper]{article}

% ---- Idioma y tipografía
\usepackage[spanish]{babel}
\usepackage[T1]{fontenc}
\usepackage[utf8]{inputenc}  % (si usas XeLaTeX/LuaLaTeX, elimínala)
\usepackage{lmodern}
\usepackage{microtype}

% ---- Márgenes y diseño
\usepackage[a4paper,margin=2.2cm]{geometry}
\usepackage{parskip}

% ---- Enlaces
\usepackage[hidelinks]{hyperref}
\usepackage{bookmark}
% ---- Cabeceras y pies
\usepackage{fancyhdr}
\pagestyle{fancy}
\fancyhf{}
\renewcommand{\headrulewidth}{0.4pt}
\cfoot{\thepage}

% ---- Listas y símbolos
\usepackage{enumitem}
\setlist{itemsep=0.25em, topsep=0.3em}

% ---- Iconos y color
\usepackage{fontawesome5}
\usepackage{xcolor}
\usepackage[most]{tcolorbox}
\tcbuselibrary{breakable,skins}

% ---- Colores propios
\definecolor{uniPrimary}{HTML}{0A7AC3}
\definecolor{uniSoft}{HTML}{E9F4FB}
\definecolor{uniAccent}{HTML}{F5B700}
\definecolor{uniOK}{HTML}{2E7D32}
\definecolor{uniWarn}{HTML}{C62828}

% ---- Metadatos reutilizables
\newcommand{\asignatura}{MULTIMEDIA}
\newcommand{\tema}{TEMA 1 — Conceptos básicos sobre multimedia e Hipermedia}  % <-- cambia el título del tema aquí
\newcommand{\clase}{Clase 1}
\newcommand{\fecha}{\today}                              % <-- pon la fecha de la clase

% ---- Cabecera con metadatos
\lhead{\textsc{\asignatura}}
\chead{\textbf{\tema}}
\rhead{\clase\;|\; \fecha}

% ---- Estética de secciones
\usepackage{titlesec}
\titleformat{\section}{\Large\bfseries\color{uniPrimary}}{\thesection}{0.5em}{}
\titleformat{\subsection}{\large\bfseries}{\thesubsection}{0.5em}{}
\titleformat{\subsubsection}{\bfseries}{\thesubsubsection}{0.5em}{}

% ---- Cajas útiles
\newtcolorbox{ObjetivosBox}{
	title={\faBullseye\; Índice},
	colback=uniSoft,
	colframe=uniPrimary,
	enhanced, breakable,
	left=8pt,right=8pt,top=8pt,bottom=8pt,
	boxrule=0.8pt,
	fonttitle=\bfseries
}

\newtcolorbox{DefBox}{
	title={\faBook\; Definición},
	colback=white,
	colframe=uniAccent,
	enhanced, breakable,
	left=8pt,right=8pt,top=8pt,bottom=8pt,
	boxrule=0.8pt,
	fonttitle=\bfseries
}

\newtcolorbox{NotaBox}{
	title={\faStickyNote\; Nota},
	colback=white,
	colframe=uniPrimary,
	enhanced, breakable,
	left=8pt,right=8pt,top=8pt,bottom=8pt,
	boxrule=0.8pt,
	fonttitle=\bfseries
}

\newtcolorbox{RecordatorioBox}{
	title={\faBell\; Recordatorio},
	colback=uniAccent!15,
	colframe=uniAccent,
	enhanced, breakable,
	left=8pt,right=8pt,top=8pt,bottom=8pt,
	boxrule=0.8pt,
	fonttitle=\bfseries
}

\newtcolorbox{ChecklistBox}{
	title={\faTasks\; Tareas / Checklist},
	colback=white,
	colframe=uniPrimary,
	enhanced, breakable,
	left=8pt,right=8pt,top=8pt,bottom=8pt,
	boxrule=0.8pt,
	fonttitle=\bfseries
}

\newtcolorbox{ResumenBox}{
	title={\faHighlighter\; Resumen rápido (5 líneas)},
	colback=uniSoft,
	colframe=uniPrimary,
	enhanced, breakable,
	left=8pt,right=8pt,top=8pt,bottom=8pt,
	boxrule=0.8pt,
	fonttitle=\bfseries
}

\newtcolorbox{VocabBox}{
	title={\faLanguage\; Vocabulario clave},
	colback=white,
	colframe=uniPrimary,
	enhanced, breakable,
	left=8pt,right=8pt,top=8pt,bottom=8pt,
	boxrule=0.8pt,
	fonttitle=\bfseries
}

% ---- Tabla estilo evaluación (por si la necesitas hoy)
\newcommand{\TablaEvaluacion}{
	\begin{center}
		\renewcommand{\arraystretch}{1.3}
		\begin{tabular}{|l|c|c|c|}
			\hline
			\textbf{Criterio} & \textbf{Porcentaje} & \textbf{Obligatorio} & \textbf{Recuperable} \\
			\hline
			Teoría (25\% cada parcial) & 50\% & Sí & Sí \\
			Práctica & 25\% & Sí & Sí \\
			Aprovechamiento en clase & 10\% & No & No \\
			Trabajo teórico & 15\% & No & Sí \\
			\hline
			\textbf{Total} & \textbf{100\%} &  &  \\
			\hline
		\end{tabular}
	\end{center}
}

% =========================================================
\begin{document}
	
	% ---- Cabecera de ficha de clase (rellenable en cada sesión)
	{\large \textbf{\asignatura} \;—\; \textbf{\tema} \hfill \textit{\clase, \fecha}}\\[0.4em]
	\faUser\; Alumno/a: Alberto Díaz\hfill
	\faChalkboardTeacher\; Profesor/a: Ana Fernández
	
	\vspace{0.6em}
	
	\tableofcontents
	
	% ---- Conceptos clave y definiciones
	\section{Definiciones}
	\begin{DefBox}
		\textbf{Multimedia:} Representación integrada de la información en forma de texto, gráficos, imágenes, vídeo.
		
		La multimedia no tiene porque ser interactiva. No necesariamente tiene que ser digital.
	\end{DefBox}
	
	
	
	\begin{DefBox}
		\textbf{Aplicación multimedia:} Software capaz de ofrecer información al usuario integrando distintos medios de representación multimedia.
		
		La información multimedia se procesa y almacena digitalmente.
	\end{DefBox}
	
	\begin{DefBox}
		\textbf{Elemento Interactivo: } Un elemento es interactivo cuando responde a la entrada de información del usuario. El usuario debe poder tomar una decisión de iniciar la interacción, de forma explicita o implícita.
	\end{DefBox}
	
	\begin{DefBox}
		\textbf{Hipertexto} Sistema de organización de datos basado en la vinculación de bloques de información llamados nodos.
	\end{DefBox}
	
	\begin{DefBox}
		\textbf{Hipermedia} Hipertexto + Multimedia. Conectar fragmentos de información que no necesariamente tienen porque ser texto si no cualquier tipo de multimedia.
	\end{DefBox}
	
	\subsection{Diferencias entre Multimedia, Hipertexto y Hipermedia}

	\begin{center}
		\renewcommand{\arraystretch}{1.4} % más espacio entre filas
		\begin{tabular}{|l|c|c|c|}
			\hline
			\textbf{Criterio} & \textbf{Multimedia} & \textbf{Hipertexto} & \textbf{Hipermedia} \\
			\hline
			\textbf{Contenido} 
			& Más de un medio
			& Solo un medio (texto) 
			& Más de un medio (texto + otros) \\
			\hline
			\textbf{Enlaces} 
			& No 
			& Sí (entre textos) 
			& Sí (entre distintos medios) \\
			\hline
			\textbf{Navegación} 
			& No (Linear)
			& Sí (No linear)
			& Sí \\
			\hline
		\end{tabular}
	\end{center}
	
	\subsection{Medios multimedia y Contexto}
	\begin{DefBox}
		\textbf{Medio} Canal que permite la distribución y comunicación de la información: Texto, gráficos, imagen, audio y vídeo.
	\end{DefBox}
	
	\subsubsection{Clasificación en función del tiempo}
	
	\begin{itemize}[leftmargin=1.5em]
		\item \textbf{Medios continuos (dependientes del tiempo):}  
		Requieren una secuencia temporal para ser comprendidos.  
		Ejemplos: \textit{audio, vídeo}.
		
		\item \textbf{Medios discretos (independientes del tiempo):}  
		No dependen de la dimensión temporal para interpretarse.  
		Ejemplos: \textit{texto, imágenes}.
	\end{itemize}
	
	\begin{NotaBox}
		\textbf{Contexto:} Esta clasificación se aplica en el ámbito de las \textbf{comunicaciones audiovisuales digitales}.
	\end{NotaBox}
	
	\subsection*{La importancia del usuario}
	\begin{DefBox}
		\textbf{Percepción} La forma en la que el cerebro interpreta los estímulos del exterior. Las percepciones evocan \textbf{sensaciones}.
	\end{DefBox}

	\begin{DefBox}
		\textbf{Sensaciones} Impresiones que los estímulos externos generan sobre las personas. Percepción de un cambio o desequilibrio.
	\end{DefBox}
	
	\begin{DefBox}
		\textbf{Emociones} Respuesta que aparece después de la sensación, alegría, tristeza...
	\end{DefBox}
	
	\textbf{Ejemplo}:  
	Veo una nube negra $\rightarrow$ \textit{percepción}.  
	Pienso que va a llover $\rightarrow$ \textit{sensación}.  
	Me invade la tristeza $\rightarrow$ \textit{emoción}.
	
	\section{Reseña Histórica}
	Primeras comunicaciones visuales surgían en la prehistoria pero no eran multimedia
	
	El invento del \textbf{transistor} fue importante para el desarrollo de la multimedia.
	
	\begin{itemize}[]
		\item \textbf{1945} Vannebar Bush propuso que los computadores deberían usarse como soporte del trabajo intelectual. Almacenamiento y comunicación de contenido multimedia y conocimiento de la humanidad. \textbf{(MEMEX) Memory Extension.}
		
		\item \textbf{1983} Se desarrolla \textbf{Intermedia}. Programación de creación hipertextual para UNIX.
		
		\item \textbf{1984} Apple lanza el \textbf{primer Macintosh} con interfaz gráfica. Primera computadora con altas capacidades de reproducción de sonidos y diseño gráfico.
		
		\item \textbf{1980´s} Aparecen \textbf{videojuegos} y software de entretenimiento.
		
		\item \textbf{1992} Es posible integrar audio, sonido  y voz, gráficas, animación de texto. Se expande la world wide web.
		
		\item \textbf{1995-2016} Gran evolución. Aparece el \textbf{metaverso}.
	\end{itemize}
	
	\subsection{Punto de inflexión: 1995-2000}
	
	\begin{itemize}[]
		\item Se extiende la señal analógica TV.
		\item \textbf{VHS}: Principal medio para grabar.
		\item Los teléfonos solo tenían llamadas y SMS.
		\item Conexión a internet es lenta. Modem.
		\item Páginas web son \textbf{estáticas}
	\end{itemize}
	
	\subsection{Punto de inflexión: 2015-2016}

	\begin{itemize}[]
		\item Señal digital de TV
		\item Teléfonos móviles son \textbf{smartphones} con muchas funciones.
		\item Páginas web son sitios interactivos, dinámicas.
		\item Mejor conexión a internet.
		\item Videollamadas.
		\item Se desarrollan las redes sociales.
	\end{itemize}
	
	\subsection{Punto de inflexión: Escena actual}

	\begin{itemize}[]
		\item Inteligencia Artificial.
		\item Segmentación de imágenes en tiempo real
		\item Realidad Virtual y Aumentada
		\item ChatGPT
	\end{itemize}

\end{document}
